% Header - packages and settings for LaTeX
% header.tex : 2010-03-24 Joram Berger

% print: linkcolors black
% online: linkcolors navy
\newif\ifprintversion
\printversionfalse

\KOMAoptions{%
	twoside=true,
	headsepline=true,
	BCOR=10mm,
	fontsize=11pt,
	listof=totoc,
	numbers=noendperiod
}
\addtokomafont{caption}{\small}  %KOMA-Skript S. 144
\addtokomafont{disposition}{\boldmath}
\setkomafont{captionlabel}{\sffamily\bfseries}
\pagestyle{headings}
\usepackage{scrhack} % make listings compatible with KOMA-script

% Packages  ---------------------------------------------------------
\usepackage[utf8]{inputenc}
\usepackage[T1]{fontenc}
\usepackage{xspace}
\usepackage[UKenglish, ngerman]{babel}% oder german %Deutsch
\usepackage[babel=true, style=american, german=guillemets]{csquotes}
\MakeAutoQuote{«}{»}
\usepackage{textcomp}
\usepackage{eurosym}
\usepackage{microtype}  % before biblatex
\usepackage{graphicx}
\graphicspath{{../fig/}}
\usepackage[intlimits]{amsmath}
\usepackage{amssymb}
\usepackage{booktabs}
\usepackage{multirow}
\usepackage{siunitx}
\usepackage{soul}

\DeclareSIUnit\clight{\ensuremath{c}}
\DeclareSIUnit[per-mode=symbol]\GeVc{\GeV\per\clight}
\DeclareSIUnit[per-mode=symbol]\GeVcc{\GeV\per\clight\squared}
\DeclareSIUnit\fb{\femto\barn}
\DeclareSIUnit\invfb{\per\femto\barn}

% vor fixmath um aufrechte Großbuchstaben zu erhalten
\let\upGamma\Gamma
\let\upTheta\Theta
\let\upLambda\Lambda
\let\upXi\Xi
\let\upPi\Pi
\let\upSigma\Sigma
\let\upUpsilon\Upsilon
\let\upPhi\Phi
\let\upPsi\Psi
\usepackage{fixmath}	% für kursive griechische Großbuchstaben
\usepackage{braket}
\usepackage{cancel}
\usepackage{slashed}
\usepackage[
style=numeric-joram,
backend=biber,
sorting=none,
date=long,
dateabbrev=true%
]{biblatex} % vor hyperref
\usepackage{tikz}
\usepackage{rotating}
\usepackage{fncychap}  % custom chapter titles
\makeatletter
	\ChNameVar{\centering\Large\sffamily}
	\ChNumVar{\Large}
	\ChTitleVar{\centering\LARGE\sffamily\bfseries}
	\ChRuleWidth{1pt}

	\renewcommand{\DOCH}{%
		\CNV\FmN{\@chapapp}\space \CNoV\thechapter
		\par\nobreak
		\vskip -0.5\baselineskip
	}
	\renewcommand{\DOTI}[1]{%
		\mghrulefill{\RW}\par\nobreak
		\CTV\FmTi{#1}\par\nobreak
		\vskip 40\p@
	}
	\renewcommand{\DOTIS}[1]{%
		\mghrulefill{\RW}\par\nobreak
		\CTV\FmTi{#1}\par\nobreak
		\vskip -10\p@
	}
\makeatother

\ifprintversion
	\definecolor{internalColor}{HTML}{000000}
	\definecolor{externalColor}{HTML}{000000}
\else
	\definecolor{internalColor}{HTML}{000080}
	\definecolor{externalColor}{HTML}{0066AA}
\fi
\definecolor{commentColor}{HTML}{666666}
\definecolor{listColor}{HTML}{EEEEEE}

% Listings  ------------------------------------------------------------------
\usepackage{listings}
\lstloadlanguages{bash,[ISO]C++,make,Python,[LaTeX]TeX}

\lstset{%language=[LaTeX]TeX,			% Sprache TeX
	numbers=left,						% Zeilennummern links
	stepnumber=1,						% jede Zeile
	numbersep=5pt,						% 5pt vom Text
	numberstyle=\tiny,					% kleine Zahlen
	breaklines=true,					% Zeilenumbruch
	breakautoindent=true, 				% mit Einrücken
	postbreak=\space, 					% bei Leerzeichen umbrechen
	tabsize=4,   						% Tabulatorlänge
	basicstyle=\footnotesize\ttfamily,	% kleine tt-Schrift für Quelltext
	keywordstyle=\bfseries\color{internalColor},% Schlüsselwörter fett (blau)
% 	identifierstyle=\underbar ,			% Bezeichner unterstreichen
	commentstyle=\color{commentColor},			% kommentar grau
	stringstyle=\color{externalColor},		% String kursiv grün
	showspaces=false,					% Leerzeichen nicht anzeigen
	showstringspaces=false,				% nur in Strings
	extendedchars=false,				% alle Zeichen anzeigen
	backgroundcolor=\color{listColor},	% Hintergrundfarbe
	captionpos=b,
	frame=lines}						% Caption unten (bottom)

\lstdefinelanguage{pythia}[ISO]{C++}{morecomment=[l]{!}}
\lstdefinelanguage{herwig}{morecomment=[l]{\#},%
morekeywords={cd,set,create,insert}}

\usepackage[printonlyused, withpage, nohyperlinks]{acronym}
%\usepackage[colorinlistoftodos, textwidth=2.6cm]{todonotes} % disable
\newcommand{\todow}[1]{\todo[color=orange!60]{#1}}
\newcommand{\todoe}[1]{\todo[color=red!80]{#1}}
\newcommand{\todoi}[1]{\todo[color=green!10]{#1}}
\usepackage[absolute]{textpos} % overlay
\usepackage[pdfa,pdfversion=1.4]{hyperref} % nach biblatex TODO: mit pdfa GANZ NACH HINTEN
\usepackage[all]{hypcap} % correct anchor point for float references
\hypersetup{
	%pdfusetitle
	pdftitle={Title},
	pdfauthor={Author},
	pdfcreator={pdfLaTeX},
	pdfsubject={Particle physics},
	pdfkeywords={CERN, LHC, CMS, particle physics},
	pdfpagemode=UseOutlines,% Inhaltsverzeichnis anzeigen beim Öffnen
	pdfdisplaydoctitle=true,% Dokumenttitel statt Dateiname anzeigen.
	pdflang=en,%Sprache des Dokuments.
	colorlinks=true,linkcolor=internalColor,citecolor=internalColor,
	filecolor=externalColor,menucolor=internalColor,urlcolor=externalColor,
	plainpages=false,%
	breaklinks=true%
%	bookmarksnumbered=true%  Überschriftsnummerierung im PDF Inhalt anzeigen.
}

% math definitions
\DeclareMathOperator\erf{erf}
\DeclareMathOperator*{\median}{median}
\newcommand{\bvec}[1]{\boldsymbol{#1}}
\newcommand{\dx}{\mathrm{d}}
\newcommand{\imag}{\mathrm{i}}
\newcommand{\euler}{\mathrm{e}}

% hep commands
\newcommand{\pT}{\ensuremath{{p_\text{T}}}\xspace}
\newcommand{\pt}[1]{\ensuremath{{p_\text{T}^\text{#1}}}}
\newcommand{\vecpt}[1]{\ensuremath{{\vec{p}_\text{T}^\text{#1}}}}
\newcommand{\pThat}[1]{\ensuremath{{\hat{p}_\text{T}^\text{#1}}}}
\newcommand{\et}[1]{\ensuremath{{E_\text{T}^\text{#1}}}}
\newcommand{\vecet}[1]{\ensuremath{{\vec{E}_\text{T}^\text{#1}}}\xspace}
\newcommand{\Eqref}[1]{Eq.~\eqref{#1}}
\newcommand{\lumi}{\ensuremath{L}}
\newcommand{\met}{\ensuremath{{E_\text{T}^\text{miss}}}\xspace}
\newcommand{\vmet}{\ensuremath{{\vec{E}_\text{T}^\text{miss}}}\xspace}
\newcommand{\alphacut}{p_\mathrm{T}^\mathrm{jet_2} / p_\mathrm{T}^\mathrm{Z}}
\newcommand{\alphacutfrac}{\frac{p_\mathrm{T}^\mathrm{jet_2}}{p_\mathrm{T}^\mathrm{Z}}}
\newcommand{\akV}{Anti-$k_\mathrm{T}~0.5$\xspace}
\newcommand{\ak}{Anti-$k_\mathrm{T}$\xspace}
\newcommand{\kt}{$k_\mathrm{T}$\xspace}
\newcommand{\pthat}{\hat{p}_{\mathrm{T}}}
\newcommand{\ndof}{\ensuremath{n_\text{dof}}}
\newcommand{\chindof}{\ensuremath{\chi^2/\ndof}}

% software etc.
\newcommand{\pythia}{\textsc{pythia}\xspace} %extra auslagern!
\newcommand{\herwig}{\textsc{herwig++}\xspace}
\newcommand{\thepeg}{{\footnotesize ThePEG}\xspace}
\newcommand{\ROOT}{\textsc{root}\xspace}
\newcommand{\matplotlib}{\textsc{matplotlib}\xspace}
\newcommand{\lhapdf}{\textsc{lhapdf}\xspace}
\newcommand{\fastjet}{\textsc{fastjet}\xspace}
\newcommand{\powheg}{\textsc{powheg}\xspace}
\newcommand{\madgraph}{\textsc{madgraph}\xspace}
\newcommand{\minlo}{\textsc{minlo}\xspace}
\newcommand{\amcnlo}{\textsc{amcnlo}\xspace}

%\DeclareUnicodeCharacter{code}{\command}
%\DeclareUnicodeCharacter{2248}{\approx}
\DeclareUnicodeCharacter{221A}{\sqrt}
%\DeclareUnicodeCharacter{222B}{\int}
%\DeclareUnicodeCharacter{2211}{\sum}
\DeclareUnicodeCharacter{2202}{\partial}
%\DeclareUnicodeCharacter{2207}{\nabla}
%\DeclareUnicodeCharacter{221E}{\infty}
\DeclareUnicodeCharacter{03B1}{\ensuremath{\alpha}}
\DeclareUnicodeCharacter{03B2}{\ensuremath{\beta}}
\DeclareUnicodeCharacter{03B3}{\ensuremath{\gamma}}
\DeclareUnicodeCharacter{03B4}{\ensuremath{\delta}}
\DeclareUnicodeCharacter{03F5}{\ensuremath{\epsilon}}
\DeclareUnicodeCharacter{03B5}{\ensuremath{\varepsilon}}
\DeclareUnicodeCharacter{03B6}{\ensuremath{\zeta}}
\DeclareUnicodeCharacter{03B7}{\ensuremath{\eta}}
\DeclareUnicodeCharacter{03B8}{\ensuremath{\theta}}
\DeclareUnicodeCharacter{03B9}{\ensuremath{\iota}}
\DeclareUnicodeCharacter{03BA}{\ensuremath{\kappa}}
\DeclareUnicodeCharacter{03BB}{\ensuremath{\lambda}}
%\DeclareUnicodeCharacter{03BC}{\ensuremath{\mu}}
\DeclareUnicodeCharacter{00B5}{\micro}
\DeclareUnicodeCharacter{03BD}{\ensuremath{\nu}}
\DeclareUnicodeCharacter{03BE}{\ensuremath{\xi}}
\DeclareUnicodeCharacter{03C0}{\ensuremath{\pi}}
\DeclareUnicodeCharacter{03C1}{\ensuremath{\varrho}}
\DeclareUnicodeCharacter{03C3}{\ensuremath{\sigma}}
\DeclareUnicodeCharacter{03C4}{\ensuremath{\tau}}
\DeclareUnicodeCharacter{03C5}{\ensuremath{\upsilon}}
\DeclareUnicodeCharacter{03C6}{\ensuremath{\varphi}}
\DeclareUnicodeCharacter{03D5}{\ensuremath{\phi}}
\DeclareUnicodeCharacter{03C7}{\ensuremath{\chi}}
\DeclareUnicodeCharacter{03C8}{\ensuremath{\psi}}
\DeclareUnicodeCharacter{03C9}{\ensuremath{\omega}}
\DeclareUnicodeCharacter{0393}{\ensuremath{\varGamma}}
\DeclareUnicodeCharacter{0394}{\ensuremath{\varDelta}}
\DeclareUnicodeCharacter{0398}{\ensuremath{\varTheta}}
%\DeclareUnicodeCharacter{039B}{\ensuremath{\varLambda}}
%\DeclareUnicodeCharacter{039E}{\ensuremath{\varXi}}
%\DeclareUnicodeCharacter{03A0}{\ensuremath{\varPi}}
%\DeclareUnicodeCharacter{03A3}{\ensuremath{\varSigma}}
%\DeclareUnicodeCharacter{03A5}{\ensuremath{\varUpsilon}}
%\DeclareUnicodeCharacter{03A6}{\ensuremath{\varPhi}}
%\DeclareUnicodeCharacter{03A8}{\ensuremath{\varPsi}}
%\DeclareUnicodeCharacter{03A9}{\ensuremath{\varOmega}}
\DeclareUnicodeCharacter{2206}{\ensuremath{\upDelta}}
\DeclareUnicodeCharacter{221D}{\ensuremath{\propto}}% ∝
\DeclareUnicodeCharacter{00B1}{\ensuremath{\pm}}    % ±
\DeclareUnicodeCharacter{00B7}{\ensuremath{\cdot}}  % ·
\DeclareUnicodeCharacter{00D7}{\ensuremath{\times}} % ×
\DeclareUnicodeCharacter{2264}{\ensuremath{\leq}}   % ≤
\DeclareUnicodeCharacter{2265}{\ensuremath{\geq}}   % ≥
\DeclareUnicodeCharacter{20AC}{\euro}
\DeclareUnicodeCharacter{2260}{\ensuremath{\neq}}
\DeclareUnicodeCharacter{2026}{\ldots}              % …
\DeclareUnicodeCharacter{00B0}{\degree}

%αβγδεζηθικλμνξοπρστυφχψωϑϕΓΔΘΛΞΠΣΥΦΨΩ ∆∂f∙∞√x∑a≈∫x– —

\hyphenation{in-for-ma-tion}

\bibliography{../bib/bibliography, ../bib/pas}

\defbibnote{bibnote}{Text before bibliography.}
